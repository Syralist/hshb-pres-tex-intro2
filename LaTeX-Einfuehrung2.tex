\documentclass[aspectratio=1610,svgnames]{beamer}

\usepackage{lmodern}
\usepackage[T1]{fontenc}
\usepackage[ngerman]{babel}
\usepackage{selinput}
\SelectInputMappings{%
   adieresis={ä},
   germandbls={ß}
   }

\usepackage{listings}
\lstset{language=[LaTeX]TeX,
escapeinside=||,
frame=single,
backgroundcolor=\color{Cornsilk},
literate={ä}{{\"a}}1
		{ö}{{\"o}}1 
		{ü}{{\"u}}1
		{Ä}{{\"A}}1
		{Ö}{{\"O}}1
		{Ü}{{\"U}}1
		{ß}{\ss}1
		{”}{''}1
		{“}{\grqq}1
		{„}{\glqq}1
}

\usetheme{PaloAlto}  %% Themenwahl

\setbeamercovered{transparent}
%\setbeamertemplate{footline}[frame number]
\usecolortheme{spruce}		% grün
\usecolortheme[named=MSUgreen]{structure}
\newcommand{\hshblogo}{\includegraphics[width=1.3cm]{space-logo}}
\newcommand{\divider}[1]{\begin{frame} %
\begin{alertblock}{} %
\centering\usebeamerfont{section title}#1 %
\end{alertblock} %
\end{frame}}
 
\title{\LaTeX{} --- Einführung Teil 2}
\author{Thomas Helmke}
\date{06.09.2014}
\logo{\includegraphics[width=1.1cm]{space-logo}}
 
\begin{document}
\maketitle
\frame{\tableofcontents}
 
\section{Aufzählungen}
\divider{\insertsection}
\subsection{Typen von Aufzählungen}
\begin{frame}[<+->] %%Eine Folie
	\frametitle{Möglichkeiten zur Aufzählung} %%Folientitel
	\begin{itemize}
		\item itemize
        \item enumerate
        \item description
	\end{itemize}
\end{frame}
\subsection{Formatierung anpassen}
\begin{frame}[<+->] %%Eine Folie
	\frametitle{Aufzählungen formatieren mit enumitem} %%Folientitel
	\begin{itemize}
		\item Paket enumitem
        \item Abstände können angepasst werden
        \item Eigene Aufzählungszeichen verwenden
	\end{itemize}
\end{frame}

\section{Floating Environments}
\divider{\insertsection}
\subsection{Bilder und Tabellen bändigen}
\begin{frame}[<+->] %%Eine Folie
	\frametitle{Bilder und Tabellen bändigen} %%Folientitel
	\begin{itemize}
		\item Paket placeins
        \item FloatBarrier
        \item Minipage
	\end{itemize}
\end{frame}

\section{Zitieren}
\divider{\insertsection}
\subsection{Wörtliche Zitate}
\begin{frame}[<+->] %%Eine Folie
	\frametitle{Wörtliche Zitate} %%Folientitel
	\begin{itemize}
        \item enquote-Befehl
		\item quote-Umgebung
        \item Zitate formatieren
	\end{itemize}
\end{frame}
\subsection{Literaturverzeichnis}
\begin{frame}[<+->] %%Eine Folie
	\frametitle{Literaturverzeichnis anlegen} %%Folientitel
	\begin{itemize}
		\item bib-Datei
        \item bibtex oder biber
        \item biblatex einbinden
	\end{itemize}
\end{frame}

\section[Natur\-wissenschaft]{Naturwissenschaft}
\divider{\insertsection}
\subsection{SI-Einheiten}
\begin{frame}[<+->] %%Eine Folie
	\frametitle{SI-Einheiten setzen} %%Folientitel
	\begin{itemize}
		\item Paket SIunitx
        \item Befehl SI
        \item Tabellen mit Einheiten
	\end{itemize}
\end{frame}
\subsection{Chemische Formeln}
\begin{frame}[<+->] %%Eine Folie
	\frametitle{Chemische Formeln setzen} %%Folientitel
	\begin{itemize}
		\item Paket mhchem
        \item Befehl ce
	\end{itemize}
\end{frame}


\end{document}
